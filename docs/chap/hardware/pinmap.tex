\section{Pinmap}

To access the onboard LEDs, buttons and switches, knowing the connected GPIO pin of the FPGA is necessary.
An overview about the accessories placed on the board is depicted in \cref{fig:floorplan_full}.
This picture also numbered all of them for later identification where the corresponding pinmap is given in the subsequent sections.

\begin{figure}[h!]
    \centering
    \input{./fig/Floorplan.pdf_tex}
    \caption{Floorplan of the MAXwel with labeled peripheral devices.}
    \label{fig:floorplan_full}
\end{figure}

\subsection{Power}
The power supply is connected to the board via a barrel jack.
A power button is placed next to the jack to turn the board on and off.
To indicate the power state, a power LED is placed on the board that always lights when the board is powered.

\subsection{Clocks}
The board has two clock sources, a $\SI{50}{MHz}$ oscillator and a $\SI{14,7456}{MHz}$ oscillator.
Both are connected to special GPIO pins that are dedicated for clock signals.
These pins provide a low skew and jitter free clock signal and provide access to the internal global clock network of the FPGA.

\begin{center}
	\begin{tabular}{c c}
		Clock & FPGA Pin \\
		\hline
		$\SI{50}{MHz}$ & PIN\_12 \\
		$\SI{14.7456}{MHz}$ & PIN\_14 \\
		\hline
	\end{tabular}
\end{center}

\subsection{LEDs}
The board has 16 LEDs, divided into two rows (namely A and B) of 8 LEDs each, that can be controlled by the FPGA.
To turn an LED on the corresponding pin has to be set to high.

\begin{center}
	\begin{tabular}{c c c}
		LED number & LED A FPGA Pin & LED B FPGA Pin \\
		\hline
		0 & PIN\_86 & PIN\_18 \\
		1 & PIN\_88 & PIN\_17 \\
		2 & PIN\_90 & PIN\_16 \\
		3 & PIN\_92 & PIN\_15 \\
		4 & PIN\_96 & PIN\_8 \\
		5 & PIN\_98 & PIN\_7 \\
		6 & PIN\_100 & PIN\_6 \\
		7 & PIN\_2 & PIN\_5 \\
		\hline
	\end{tabular}
\end{center}

\subsection{Buttons and Switches}
The board is equipped with 4 buttons and 8 switches.
The buttons are debounced with a $\SI{8}{kHz}$ low pass filter and should drive the input pin to high when pressed.

\begin{center}
	\begin{tabular}{c c}
		Button number & FPGA Pin \\
		\hline
		0 & PIN\_3 \\
		1 & PIN\_4 \\
		2 & PIN\_19 \\
		3 & PIN\_20 \\
		\hline
	\end{tabular}
\end{center}

The switches are also debounced with a $\SI{8}{kHz}$ low pass filter and should drive the input pin to high when the switch is in the upper position.

\begin{center}
	\begin{tabular}{c c}
		Switch number & FPGA Pin \\
		\hline
		0 & PIN\_85 \\
		1 & PIN\_87 \\
		2 & PIN\_89 \\
		3 & PIN\_91 \\
		4 & PIN\_95 \\
		5 & PIN\_97 \\
		6 & PIN\_99 \\
		7 & PIN\_1 \\
		\hline
	\end{tabular}
\end{center}

\subsection{Seven Segment Display}
The MAXwel is equipped with a four digit seven segment display.
The seven segment displays have a common anode: to turn on a segment the corresponding pin has to be set to high.
The labels for the segments are depicted in \cref{fig:seven_segment_display}.

\begin{figure}[h!]
    \centering
    \input{./fig/sevensegment.pdf_tex}
    \caption{Seven segment display with labeled segments.}
    \label{fig:seven_segment_display}
\end{figure}

\begin{center}
	\begin{tabular}{c c c c c}
		Segment Name & Segment 0 & Segment 1 & Segment 2 & Segment 3 \\
		\hline
		A & PIN\_56 & PIN\_48 & PIN\_38 & PIN\_28 \\
		B & PIN\_54 & PIN\_44 & PIN\_36 & PIN\_26 \\
		C & PIN\_55 & PIN\_47 & PIN\_37 & PIN\_27 \\
		D & PIN\_58 & PIN\_50 & PIN\_40 & PIN\_30 \\
		E & PIN\_62 & PIN\_52 & PIN\_42 & PIN\_34 \\
		F & PIN\_57 & PIN\_49 & PIN\_39 & PIN\_29 \\
		G & PIN\_61 & PIN\_51 & PIN\_41 & PIN\_33 \\
		DP & PIN\_53 & PIN\_43 & PIN\_35 & PIN\_21 \\
		\hline
	\end{tabular}
\end{center}

\subsection{External pin header}
The orientation of the external pin header is determined by the reference triangle.
All Pins on the pin header are connected directly to the GPIO pins of the FPGA so take care when connecting external devices to the board.
Do not exceed the voltage range of $\SI{0}{V}$ to $\SI{3.3}{V}$ and do not draw too much current from the pins.
The pin header is depicted in \cref{fig:external_pin_header}.

\begin{figure}[h!]
    \centering
    \input{./fig/Pinheader.pdf_tex}
    \caption{External pin header with labeled pins.}
    \label{fig:external_pin_header}
\end{figure}

\subsection{Programming header}
The programming header is used to program the FPGA with a JTAG programmer.
The pinout of the programming header is depicted in \cref{fig:programming_header}.
The \textit{TCK} pin is connected to a $\SI{10}{k\Omega}$ pull down resistor and the \textit{TDI}, \textit{TMS} and \textit{TDO} pins are connected to a $\SI{10}{k\Omega}$ pull up resistor.

\begin{figure}[h!]
    \centering
    \input{./fig/Progheader.pdf_tex}
    \caption{Programming header with labeled pins.}
    \label{fig:programming_header}
\end{figure}
