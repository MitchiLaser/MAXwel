\section{Pinmap}

To access the onboard LEDs, buttons and switches, knowing the connected GPIO pin of the FPGA is necessary.
An overview about the accessories placed on the board is depicted in \cref{fig:floorplan_full}.
This picture also numbered all of them for later identification where the corresponding pinmap is given in the subsequent sections.

\begin{figure}[h!]
    \centering
    \input{./fig/Floorplan.pdf_tex}
    \caption{Floorplan of the MAXwel with labeled peripheral devices.}
    \label{fig:floorplan_full}
\end{figure}

\subsection{Power}
The power supply is connected to the board via a barrel jack.
A power button is placed next to the jack to turn the board on and off.
To indicate the power state, a power LED is placed on the board that always lights when the board is powered.

\subsection{Clocks}
The board has two clock sources, a $\SI{50}{MHz}$ oscillator and a $\SI{14,7456}{MHz}$ oscillator.
Both are connected to special GPIO pins that are dedicated for clock signals.
These pins provide a low skew and jitter free clock signal and provide access to the internal global clock network of the FPGA.

\begin{center}
	\begin{tabular}{c c}
		Clock & FPGA Pin \\
		\hline
		$\SI{50}{MHz}$ & PIN\_12 \\
		$\SI{14.7456}{MHz}$ & PIN\_14 \\
		\hline
	\end{tabular}
\end{center}

\subsection{LEDs}
The board has 16 LEDs, 8 green and 8 red, that can be controlled by the FPGA.
To turn an LED on the corresponding pin has to be set to high.
\todo[inline]{Add pinmap for LEDs.}

\begin{center}
	\begin{tabular}{c c c}
		LED number & LED Green FPGA Pin & LED Red FPGA Pin \\
		\hline
		1 & PIN\_ & PIN\_ \\
		2 & PIN\_ & PIN\_ \\
		3 & PIN\_ & PIN\_ \\
		4 & PIN\_ & PIN\_ \\
		5 & PIN\_ & PIN\_ \\
		6 & PIN\_ & PIN\_ \\
		7 & PIN\_ & PIN\_ \\
		8 & PIN\_ & PIN\_ \\
		\hline
	\end{tabular}
\end{center}

\subsection{Buttons and Switches}
The board is equipped with 4 buttons and 8 switches.
The buttons are debounced with a $\SI{8}{kHz}$ low pass filter and should drive the input pin to high when pressed.
\todo[inline]{Add pinmap for buttons.}

\begin{center}
	\begin{tabular}{c c}
		Button number & FPGA Pin \\
		\hline
		1 & PIN\_ \\
		2 & PIN\_ \\
		3 & PIN\_ \\
		4 & PIN\_ \\
		\hline
	\end{tabular}
\end{center}

The switches are also debounced with a $\SI{8}{kHz}$ low pass filter and should drive the input pin to high when the switch is in the upper position.
\todo[inline]{Add pinmap for switches and checn information about position.}

\begin{center}
	\begin{tabular}{c c}
		Switch number & FPGA Pin \\
		\hline
		1 & PIN\_ \\
		2 & PIN\_ \\
		3 & PIN\_ \\
		4 & PIN\_ \\
		5 & PIN\_ \\
		6 & PIN\_ \\
		7 & PIN\_ \\
		8 & PIN\_ \\
		\hline
	\end{tabular}
\end{center}

\subsection{Seven Segment Display}
All 8 LEDs within the seven segment display are turned on by setting the corresponding pin to high.
\todo[inline]{Add pinmap for seven segment display.}
\todo[inline]{Add picture of corresponding LEDs in the seven segment display.}

\begin{center}
	\begin{tabular}{c c c c c}
		Segment LED & Segment 1 & Segment 2 & Segment 3 & Segment 4 \\
		\hline
		A & PIN\_ & PIN\_ & PIN\_ & PIN\_ \\
		B & PIN\_ & PIN\_ & PIN\_ & PIN\_ \\
		C & PIN\_ & PIN\_ & PIN\_ & PIN\_ \\
		D & PIN\_ & PIN\_ & PIN\_ & PIN\_ \\
		E & PIN\_ & PIN\_ & PIN\_ & PIN\_ \\
		F & PIN\_ & PIN\_ & PIN\_ & PIN\_ \\
		G & PIN\_ & PIN\_ & PIN\_ & PIN\_ \\
		DP & PIN\_ & PIN\_ & PIN\_ & PIN\_ \\
		\hline
	\end{tabular}
\end{center}


\subsection{External Pinheader}
\todo[inline]{Add pinmap for external pinheader (picture).}

\subsection{Programming header}
\todo[inline]{Add pinmap for programming header (picture).}
\todo[inline]{Add information about pull up and down resistors}
